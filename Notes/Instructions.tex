%\begin{preamble}
	\documentclass[12pt]{article}
	\usepackage{amsthm}
	\usepackage{amsmath}
	\usepackage{amssymb}
	\usepackage{mathabx}
	\usepackage{stmaryrd}
	\newcommand{\sv}[1]{\ensuremath{\llbracket #1 \rrbracket}}
	\usepackage[margin=1in]{geometry}
	\usepackage[normalem]{ulem}
	\usepackage{hyperref}
	\usepackage{tipa}
	\usepackage{gb4e}
	\noautomath
	\usepackage{enumerate}
	\usepackage{colortbl}
	\usepackage{multicol}
	\usepackage{palatino}
	\usepackage{graphicx}
	\usepackage{setspace}
	\usepackage{titlesec}
	\usepackage{framed}
	\usepackage[english]{babel}
	\usepackage{natbib}
	\usepackage{color}
	\usepackage[font=small,labelfont=bf]{caption}
	\usepackage{wrapfig}
	\usepackage{pifont}
 	\newcommand{\hand}{\ding{43}}
	\synctex=1
	\setlength{\parindent}{3em}
%\end{preamble}

\begin{document}

\Large
\noindent \textbf{Instructions for changing stimuli}\\
\normalsize
\noindent Morgan Moyer \\
\noindent Edit: 10/23/17 \\

\bigskip
\noindent \textbf{Goal}: to substitute all the embedding verbs from the four studies we currently have running with either \emph{X told Y} or \emph{X informed Y}, but make as minimal other changes as possible.\\

\noindent We have 2 studies and 4 blocks per study, so 8 files in total that need to be modified. Concretely, there are two modifications. The first is simple, the second is slightly more complex:
\begin{exe}
\ex Change the final line (e.g., ``Is John right?'') to ``Is that right?''
\ex Substitute \emph{know} and \emph{predict} with either \emph{X told Y} or \emph{X informed Y}.\\
\textbf{Notes}:
	\begin{xlist}
	\ex I said this is ``complicated'' because I need half the items to be one verb and half to be the other verb (8 and 8). 
	\ex BUT, of the two verbs, \emph{X told Y} can be pretty much substituted across the board for \emph{know} or \emph{predict}, while \emph{X informed Y} seems to be more constrained in which scenarios it can be felicitously used.
	\ex SO. First try to find stories that are compatible (meaning, don't sound weird) with \emph{informed}. Then fill in the rest with \emph{tell}.
	\ex ANOTHER NOTE: for a given story, you need to be consistent in the verb you repace across other experimental files. I.E. if you use `tell' in the `finding dog toys' story for the `Blue' study file, then you need to do it for all the other files as well.
	\end{xlist}
\end{exe}

\bigskip
\large
\noindent \textbf{Example}\\
\normalsize

\begin{exe}
\ex \textbf{Old Stimuli}\\
Tim and Jen needs some advice about what they should do after high school because they're not sure about going to college just yet. They has several friends who like to give advice: Jim, Kate, Fran, Jess, and Connie. \\

Kate and Fran are discussing this. Kate asks Fran, "Who do you think they will ask?" \\

Fran says, "I think Connie and Jess." \\

It turns out that Tim and Jen ask Jim, Kate and Fran, but not Jess or Connie.\\

Kate reports, "Fran knows who they asked for advice." \\

Is Kate right?

\ex \textbf{New Stimuli}\\
Tim and Jen needs some advice about what they should do after high school because they're not sure about going to college just yet. They has several friends who like to give advice: Jim, Kate, Fran, Jess, and Connie. \\

Kate and Fran are discussing this. Kate asks Fran, "Who do you think they will ask?" \\

Fran says, "I think Connie and Jess." \\

It turns out that Tim and Jen ask Jim, Kate and Fran, but not Jess or Connie.\\

\textbf{CHANGE}: Fran \emph{told/informed} Kate who they asked for advice. \\

\textbf{CHANGE}: Is that right?

\end{exe}

\end{document}